\documentclass[11pt]{scrartcl}
\usepackage{amsmath,amssymb,amsthm}
\usepackage{graphicx}
\usepackage[usenames,dvipsnames,svgnames]{xcolor}
\usepackage{hyperref}
\usepackage{mathtools}
\usepackage{microtype}
\usepackage{xcolor}
\usepackage{asymptote}

%%% Macros
\providecommand{\ol}{\overline}
\providecommand{\ul}{\underline}
\providecommand{\wt}{\widetilde}
\providecommand{\wh}{\widehat}
\providecommand{\eps}{\varepsilon}
\providecommand{\half}{\frac{1}{2}}
\providecommand{\inv}{^{-1}}
\newcommand{\dang}{\measuredangle} %% Directed angle
\providecommand{\CC}{\mathbb C}
\providecommand{\FF}{\mathbb F}
\providecommand{\NN}{\mathbb N}
\providecommand{\QQ}{\mathbb Q}
\providecommand{\RR}{\mathbb R}
\providecommand{\ZZ}{\mathbb Z}
\providecommand{\ts}{\textsuperscript}
\providecommand{\dg}{^\circ}
\providecommand{\ii}{\item}
\providecommand{\defeq}{\coloneqq}
\DeclareMathOperator*{\lcm}{lcm}
\DeclareMathOperator*{\argmin}{arg min}
\DeclareMathOperator*{\argmax}{arg max}
\providecommand{\hrulebar}{\par
\hspace{\fill}\rule{0.95\linewidth}{.7pt}\hspace{\fill}
\par\nointerlineskip \vspace{\baselineskip}}

%%% Colored sections
\renewcommand*{\sectionformat}%
	{\color{purple}\S\thesection\enskip}
	\renewcommand*{\subsectionformat}%
	{\color{purple}\S\thesubsection\enskip}
	\renewcommand*{\subsubsectionformat}%
	{\color{purple}\S\thesubsubsection\enskip}
	\KOMAoptions{numbers=noenddot}

%%% Optional colored theorems
\usepackage{thmtools}
\usepackage[framemethod=TikZ]{mdframed}
\mdfdefinestyle{mdbluebox}{%
	roundcorner = 10pt,
	linewidth=1pt,
	skipabove=12pt,
	innerbottommargin=9pt,
	skipbelow=2pt,
	linecolor=blue,
	nobreak=true,
	backgroundcolor=TealBlue!5,
}
\declaretheoremstyle[
	headfont=\sffamily\bfseries\color{MidnightBlue},
	mdframed={style=mdbluebox},
	headpunct={\\[3pt]},
	postheadspace={0pt}
]{thmbluebox}
\mdfdefinestyle{mdredbox}{%
	linewidth=0.5pt,
	skipabove=12pt,
	frametitleaboveskip=5pt,
	frametitlebelowskip=0pt,
	skipbelow=2pt,
	frametitlefont=\bfseries,
	innertopmargin=4pt,
	innerbottommargin=8pt,
	nobreak=true,
	backgroundcolor=Salmon!5,
	linecolor=RawSienna,
}
\declaretheoremstyle[
	headfont=\bfseries\color{RawSienna},
	mdframed={style=mdredbox},
	headpunct={\\[3pt]},
	postheadspace={0pt},
]{thmredbox}
\mdfdefinestyle{mdgreenbox}{%
	skipabove=8pt,
	linewidth=2pt,
	rightline=false,
	leftline=true,
	topline=false,
	bottomline=false,
	linecolor=ForestGreen,
	backgroundcolor=ForestGreen!5,
}
\declaretheoremstyle[
	headfont=\bfseries\sffamily\color{ForestGreen!70!black},
	bodyfont=\normalfont,
	spaceabove=2pt,
	spacebelow=1pt,
	mdframed={style=mdgreenbox},
	headpunct={ --- },
]{thmgreenbox}
\mdfdefinestyle{mdblackbox}{%
	skipabove=8pt,
	linewidth=3pt,
	rightline=false,
	leftline=true,
	topline=false,
	bottomline=false,
	linecolor=black,
	backgroundcolor=RedViolet!5!gray!5,
}
\declaretheoremstyle[
	headfont=\bfseries,
	bodyfont=\normalfont\small,
	spaceabove=0pt,
	spacebelow=0pt,
	mdframed={style=mdblackbox}
]{thmblackbox}

\declaretheorem[style=thmbluebox,name=Theorem,numberwithin=section]{theorem}
\declaretheorem[style=thmbluebox,name=Lemma,sibling=theorem]{lemma}
\declaretheorem[style=thmbluebox,name=Proposition,sibling=theorem]{proposition}
\declaretheorem[style=thmbluebox,name=Corollary,sibling=theorem]{corollary}
\declaretheorem[style=thmbluebox,name=Theorem,numbered=no]{theorem*}
\declaretheorem[style=thmbluebox,name=Lemma,numbered=no]{lemma*}
\declaretheorem[style=thmbluebox,name=Proposition,numbered=no]{proposition*}
\declaretheorem[style=thmbluebox,name=Corollary,numbered=no]{corollary*}
\declaretheorem[style=thmgreenbox,name=Algorithm,sibling=theorem]{algorithm}
\declaretheorem[style=thmgreenbox,name=Algorithm,numbered=no]{algorithm*}
\declaretheorem[style=thmgreenbox,name=Claim,sibling=theorem]{claim}
\declaretheorem[style=thmgreenbox,name=Claim,numbered=no]{claim*}
\declaretheorem[style=thmredbox,name=Example,sibling=theorem]{example}
\declaretheorem[style=thmredbox,name=Example,numbered=no]{example*}
\declaretheorem[style=thmblackbox,name=Remark,sibling=theorem]{remark}
\declaretheorem[style=thmblackbox,name=Remark,numbered=no]{remark*}

\begin{asydef}
	defaultpen(fontsize(10pt));
	size(8cm); // set a reasonable default
	usepackage("amsmath");
	usepackage("amssymb");
	settings.tex="pdflatex";
	settings.outformat="pdf";
	// Replacement for olympiad+cse5 which is not standard
	import geometry;
	// recalibrate fill and filldraw for conics
	void filldraw(picture pic = currentpicture, conic g, pen fillpen=defaultpen, pen drawpen=defaultpen)
		{ filldraw(pic, (path) g, fillpen, drawpen); }
	void fill(picture pic = currentpicture, conic g, pen p=defaultpen)
		{ filldraw(pic, (path) g, p); }
	// some geometry
	pair foot(pair P, pair A, pair B) { return foot(triangle(A,B,P).VC); }
	pair orthocenter(pair A, pair B, pair C) { return orthocentercenter(A,B,C); }
	pair centroid(pair A, pair B, pair C) { return (A+B+C)/3; }
	// cse5 abbrevations
	path CP(pair P, pair A) { return circle(P, abs(A-P)); }
	path CR(pair P, real r) { return circle(P, r); }
	pair IP(path p, path q) { return intersectionpoints(p,q)[0]; }
	pair OP(path p, path q) { return intersectionpoints(p,q)[1]; }
	path Line(pair A, pair B, real a=0.6, real b=a) { return (a*(A-B)+A)--(b*(B-A)+B); }
	// cse5 more useful functions
	picture CC() {
		picture p=rotate(0)*currentpicture;
		currentpicture.erase();
		return p;
	}
	pair MP(Label s, pair A, pair B = plain.S, pen p = defaultpen) {
		Label L = s;
		L.s = "$"+s.s+"$";
		label(L, A, B, p);
		return A;
	}
	pair Drawing(Label s = "", pair A, pair B = plain.S, pen p = defaultpen) {
		dot(MP(s, A, B, p), p);
		return A;
	}
	path Drawing(path g, pen p = defaultpen, arrowbar ar = None) {
		draw(g, p, ar);
		return g;
	}
\end{asydef}


\begin{document}
\title{Complete solutions to the problems}
\subtitle{Only for people who actually care about the proofs}
\author{The IMO Shortlist}
\date{41st Mystery Hunt 2021}
\maketitle

\section{Algebra}
\subsection*{Solution A1}
This is a quadratic equation in $y$, so we may simply expand to get
\[ 9y^2 + (24x - 12(4x+1)) y + 16x^2+4(4x+1) = 0 \]
which factors as
\[ \left( 3y - (4x+2) \right)^2 = 0. \]
Hence, the solution is $y = \frac{4x+2}{3}$.

\subsection*{Solution A2}
We let $A > 0$ denote the desired expression.
Square both sides to obtain
\begin{align*}
	A^2 &= \left( 20x+3+4\sqrt{15x} \right)
	+ \left( 20x+3-4\sqrt{15x} \right)
	+ 2\sqrt{ (20x+3)^2-(4\sqrt{15}x)^2 } \\
	&= 40x+6 + 2\sqrt{400x^2 - 120x + 9} \\
	&= 40x+6 + 2(20x-3) = 80x.
\end{align*}
Therefore, $A = \sqrt{80x}$.

\subsection*{Solution A3}
Let $g(x) = \frac{f(x)+1}{3}$.
Then the condition means we have the convolution
\[ g \ast \mathbf 1 = \operatorname{id} \]
where $\mathbf 1$ is the constant function $1$.
This means that $g$ coincides with the Euler phi function $\varphi$.
Hence, it follows that \[ f(x) = 3 \varphi(x) - 1. \]

\subsection*{Solution A4}
Start with the identity
\[ \sum_{n \ge 0} T^n = \frac{1}{1-T} \]
valid for $|T| < 1$.
Differentiate both sides to obtain
$\sum_{n \ge 0} n T^{n-1} = \frac{1}{(1-T)^2}$
or equivalently
\[ \sum_{n \ge 0} n T^{n} = \frac{T}{(1-T)^2}. \]
Consequently, if we choose $T = \frac{1}{1+x^{-1/2}}$, we obtain
\begin{align*}
	\sum_{n \ge 0} n \left( 1 + \frac{1}{\sqrt x} \right)^{-n}
	&= \frac{\frac{1}{1+\frac{1}{\sqrt x}}}
	{\left( 1 - \frac{1}{1+\frac{1}{\sqrt x}} \right)^2} \\
	&= \frac{1+\frac{1}{\sqrt x}}
	{\left( \left( 1 + \frac{1}{\sqrt x} \right) - 1 \right)^2} \\
	&= x + \sqrt x.
\end{align*}

\subsection*{Solution A5}
This problem is taken from Putnam 2008 B3 which has the following form:
\begin{claim*}
	The largest possible radius of a circle inside
	an $n$-dimensional hypercube of side length $s > 0$
	is exactly $\frac s2 \sqrt{\frac n2}$.
\end{claim*}
\begin{proof}
	By scaling, we assume without loss of generality that $s = 2$.
	By symmetry, we may assume the center is zero.

	Parametrize the circle as
	\[ \mathbf x(t)
		= \left( \mathbf a \cos t
		+ \mathbf b \sin t \right) \]
	where vectors 
	\begin{align*}
		\mathbf a &= (a_1, \dots, a_n) \\
		\mathbf b &= (b_1, \dots, b_n)
	\end{align*}
	are orthogonal of the same length $r$.
	We require that every component is contained inside $[-1,1]$
	across all $t \in \RR$.

	For the upper bound, note that we thus require
	$\sqrt{a_i^2+b_i^2} \le 1$ for every $i$,
	whence squaring and summing gives
	$n \ge \sum_i(a_i^2+b_i^2) = 2r^2$.

	For the lower bound, we need two constructions.
	\begin{itemize}
		\ii For $n \ge 2$ odd
		$\mathbf a = (1,\dots,1,0,\dots,0)$
		and $\mathbf b = (0,\dots,0,1,\dots,1)$.
		\ii For $n \ge 3$ odd, we can set
		\begin{align*}
			\mathbf a &= \left( \frac{\sqrt3}{2}, \frac{-\sqrt3}{2}, 0 \right) \\
			\mathbf b &= \left(\half, \half, -1\right)
		\end{align*}
		for $n=3$ and then inductively add two components. \qedhere
	\end{itemize}
\end{proof}
Applying this to the given problem, we find the necessary
and sufficient condition $d$ is that
\[ \sqrt x \le \frac{\pi}{2} \sqrt{\frac{d}{2}}. \]

Put another way, we have $d \ge \frac{8}{\pi^2} x$.
Since $d$ must be an integer, we conclude
\[ d \ge \left\lceil \frac{8}{\pi^2} x \right\rceil. \]

\subsection*{Solution A6}
We rewrite the given equation as
\[ 2f(x+y) - 1 = 4f(x)f(y) - 2f(x) - 2f(y) + 1
	= \left( 2f(x)-1 \right)\left( 2f(y)-1 \right). \]
Letting $g(x) = 2f(x)-1$, we find $g(x+y) = g(x)g(y)$.

Note that if any value of $g$ is equal to zero, then the function $g$ is identically zero,
which is not a valid solution since we assumed the function $f$ was strictly increasing.
So in what follows we assume $g$ is never zero.

Since $g(x) = g(x/2)^2 > 0$ for any $x$,
it follows that $\log g$ is a well-defined additive function
(meaning $\log g(x+y) = \log g(x) + \log g(y)$)
which is also strictly increasing.
It is a property of Cauchy's equation that this implies $\log g$ is a linear function.
Consequently, we conclude that $g$ is an exponential function.

Putting this all together, we derive that
\[ f(x) = \frac{c^x+1}{2} \]
for some constant $c$.
Since we need $f(24) = 5$, it follows that the correct constant is $c = 9^{1/24}$.
In other  words,
\[ f(x) = \frac{9^{x/24} + 1}{2} \]
is the only solution to the functional equation.


\section{Combinatorics}
\subsection*{Solution C1}
There are $\binom{2n}{n}$ ways to choose $A \cup B$
after which there are $2^n$ ways to split $A \cup B$ into sets $A$ and $B$.
In other words, we find
\[ M = 2^n \binom{2n}{n}. \]
By Legendre's formula, the exponent of $2$ in $n!$ is $n - s_2(n)$,
while the exponent of $2$ in $(2n)!$ is $2n - s_2(2n) = 2n - s_2(n)$,
where $s_2(n)$ is the number of $1$'s in the binary representation of $n$.
So the exponent of $2$ in $\binom{2n}{n}$ is
exactly $[2n-s_2(n)] - 2[n-s_2(n)] = s_2(n)$.

From this we conclude that
\[ e = n + s_2(n) \]
is the final answer.

\subsection*{Solution C2}
This problem is adapted from USAMO 2009 problem 2
and the answer is $2 \left\lceil n/2 \right\rceil$.

By shifting, we will work on the analogous problem
where we want a subset of $\{-n, -(n-1), \dots, 0, \dots, n \}$
and we want no three elements to have sum equal to zero.

To show $2 \left\lceil n/2 \right\rceil$ is achievable
it now suffices to take all odd numbers.

We now turn our attention to showing this is best possible.
To prove this is maximal, it suffices to show it for $n$ even;
we do so by induction on even $n \ge 2$ with the base case being trivial.
Letting $A$ be the subset, we consider three cases:
\begin{enumerate}
	\item If $|A \cap \{-n,-n+1,n-1,n\}| \le 2$,
	then by the hypothesis for $n-2$ we are done.
	\item If both $n \in A$ and $-n \in A$,
	then there can be at most $n-2$ elements in $A \setminus \{\pm n\}$,
	one from each of the pairs $(1,n-1)$, $(2,n-2)$, $\dots$
	and their negations.
	\item If $n, n-1, -n+1 \in A$ and $-n \notin A$,
	then at most $n-3$ more can be added,
	one from each of $(1, n-2)$, $(2, n-3)$, $\dots$
	and $(-2, -n+2)$, $(-3, -n+3)$, $\dots$.
	(In particular $-1 \notin A$.
	Analogous case for $-A$ if $n \notin A$.)
\end{enumerate}
Thus in all cases, $|A| \le n$ as needed.

\subsection*{Solution C3}
This problem is based on China TST 2018, test 2, problem 3.

The answer to the given problem is $\frac{n+2014}{\gcd(n,2014)}$.
In general, we replace $2014$ by $m$ and show the answer
$\frac{n+m}{\gcd(m,n)}$.

Note that if $d = \gcd(m,n) > 1$, we could focus only on
the numbers which are $0 \pmod d$, which is the same process
all numbers scaled by $d$ and the pair $(m,n)$ replaced by the pair $(m/d, n/d)$.
The same logic applies to numbers which are $1 \pmod d$, etc.
Thus we may scale down by a factor of $d$ without any loss of generality.
So in what follows, we will always assume $\gcd(m,n) = 1$,
and show the answer is $m+n$.

To see that the operation could go on indefinitely,
we notice that writing $\{1, \dots, m\}$ and $\{1, \dots, n\}$ is sufficient.

Now we prove that fewer than $m+n$ numbers does not suffice.
In fact it suffices to work modulo $m+n$.
We consider the finite simple undirected graph $G$
defined on the vertex set $V = \ZZ/(m+n)$ with edges
$x \to x+m$ and $x \to x+n$ for each $x \in \ZZ/(m+n)$.
(These are equivalent to $x \to x-n$
and $x \to x-m$ modulo $m+n$, respectively.)
Since $\gcd(m,n) = 1$, this graph is connected and has $m+n$ edges
(actually, it is isomorphic to a cycle on $m+n$ vertices).

For every number on the board, we place a chip at the corresponding residue in $V$
and then observe the given operation corresponds to chip-firing.
It is a standard lemma from chip-firing that the number of chips
needed for chip-firing to continue indefinitely is at least the number of edges of $G$,
which is exactly $m+n$.

\subsection*{Solution C4}
This problem is adapted from USA Team Selection Test for IMO 2015, problem 5.

First, we eliminate the distractions of coloring the vertices from the problem:
\begin{claim*}
	The vertices are pairwise distinct colors.
	Moreover, no color used on an edge can be used on a vertex.
\end{claim*}
\begin{proof}
	It suffices to verify this condition
	among any sub-tournament with only three vertices
	(since given an edge $e$ and a vertex $v$ we can find three vertices
	containing both endpoints of $e$ and the vertex $v$).

	Now among any three vertices there is always a way to label them $a$, $b$, $c$
	such that $a \to b$ and $b \to c$,
	and so the conclusions of the claim follows.
\end{proof}

Now suppose there are $\chi$ colors represented among
the edges.

\begin{claim*}
	The minimum possible value of $\chi$ is $\left\lceil \log_2 n \right\rceil$.
\end{claim*}
\begin{proof}
	First, we prove by induction on $n$
	that $\chi \ge \log_2 n$ for any coloring and any tournament.
	The base case $n = 1$ is obvious.
	Now given any tournament, consider any used color $c$.
	Then it should be possible to divide the tournament
	into two subsets $A$ and $B$ such that all $c$-colored edges
	point from $A$ to $B$
	(for example by letting $A$ be all vertices
	which are the starting point of a $c$-edge).
	\begin{center}
	\begin{asy}
		size(5cm);
		filldraw(box( (-2,-2), (-0.6,2) ), opacity(0.1)+lightcyan, lightblue );
		filldraw(box( (0.6,-2), (2,2) ), opacity(0.1)+lightcyan, lightblue );
		label("$A$", (-2,0), dir(180), blue);
		label("$B$", ( 2,0), dir(  0), blue);
		label("all edges colored $c$", (0,2), dir(90), deepgreen);
		draw( (-1.3,1.2)--(1.5,0.2), deepgreen, EndArrow);
		draw( (-0.9,-0.2)--(1.1, -0.1), deepgreen, EndArrow);
		draw( (-1.4, 0.4)--(0.9, 0.8), deepgreen, EndArrow);
		draw( (-1.4, -0.8)--(1.3, -0.6), deepgreen, EndArrow);
		draw( (-1.3, -0.5)--(1.2, -1.6), deepgreen, EndArrow);
	\end{asy}
	\end{center}
	One of $A$ and $B$ has size at least $n/2$, say $A$.
	Since $A$ has no $c$ edges,
	and uses at least $\log_2 |A|$ colors other than $c$, we get
	\[ \chi \ge 1 + \log_2 (n/2) = \log_2 n \]
	completing the induction.

	One can read the construction off from the argument above,
	but here is a concrete description.
	For each integer $n$,
	consider the tournament whose vertices are
	the binary representations of $S  =\{0, \dots, n-1\}$.
	Instantiate colors $c_1$, $c_2$, \dots.
	Then for $v, w \in S$,
	we look at the smallest order bit for which they differ;
	say the $k$th one.
	If $v$ has a zero in the $k$th bit,
	and $w$ has a one in the $k$th bit,
	we draw $v \to w$.
	Moreover we color the edge with color $c_k$.
	This works and uses at most $\left\lceil \log_2 n \right\rceil$ colors.
\end{proof}

Hence, the final answer to the question
is $n + \left\lceil \log_2 n \right\rceil$.



\subsection*{Solution C5}
For $n > 6$ we claim the answer is $n+6$.
This is adapted from Tournament of Towns, Spring 2016, Problem J-A7.

This is a consequence of Tur\'{a}n's theorem.
View the cards as vertices of a graph $G$,
and the pointed cards as edges of the graph $G$.
Happiness is guaranteed if and only if it is impossible
to find an \emph{independent set} with at least $n/2$ cards.

Tur\'{a}n's theorem then implies that the graph $G$ consisting of
$\frac{n-6}{2}$ copies of $K_2$ and $2$ copies of $K_3$,
which obviously has independence number
$\frac{n-6}{2} + 2 = \frac n2 - 1$, is best possible.

\subsection*{Solution C6}
Let $\ell = 2n$. We claim that $\ell \ge 2m-2$ for $m \ge 3$
and that this is best possible.

We will only deal with the case $m \ge 3$.
For the construction, the case $m=3$ is a famous brainteaser (shown below),
and then one can inductively add a vertical/horizontal
line in order to to get $\ell = 2m-2$ for any $m \ge 4$.
\begin{center}
\begin{asy}
	size(5cm);
	dotfactor *= 2;
	draw( (3,1.5)--(3,4)--(0,1)--(3,1)--(0.5,3.5), red );
	for (int i=1; i<=3; ++i) {
	for (int j=1; j<=3; ++j) {
		dot( (i,j) );
	}
	}
\end{asy}
\end{center}

Now, we show $\ell \ge 2m-2$.
Assume there are $m-a$ horizontal lines
and $m-b$ vertical lines.
We first dispense of some edge cases:
\begin{itemize}
	\ii If $a \le 0$,
	then there are at least $m$ horizontal lines,
	which must be joined by at least
	$m-1$ other lines, so $\ell \ge 2m-1$.
	\ii If $a = 1$, then there is
	some row of points is untouched by the horizontal lines,
	and at least $m$ non-horizontal lines are needed to
	pass through these.
	So $\ell \ge (m-1) + m = 2m-1$.
\end{itemize}
The cases $b \le 0$ and $b = 1$ are handled analogously.
Hence in what follows assume $\min(a,b) \ge 2$.

Then we have an $a \times b$ sub-grid of points
(perhaps not evenly spaced, but still rectangular)
not touched by any horizontal or vertical lines.
Consider the boundary of this grid, which has $2a+2b-4$ points.
Each line passes through at most $2$ points on this boundary.
So we need at least $a+b-2$ lines not horizontal or vertical.
In conclusion,
\[ \ell \ge (a+b-2) + (m-a) + (m-b) = 2m-2. \]
Since $\ell = 2n$, we get $n \ge m-1$ as needed.

Put another way, we have $m \le n+1$ as the answer.

\section{Geometry}
\subsection*{Solution G1}
The hidden collinearity is that $U$, $N$, $K$ are collinear, which we prove now.

It is a classical lemma that since $T$ is the arc midpoint of $\widehat{SN}$,
the point $U$ is the incenter of $\triangle CSN$.
Consequently, since $A$ is the arc midpoint of $\widehat{SK}$,
it follows that ray $SA$ is the angle bisector of $\angle CS$,
so it passes through the point $U$.
\begin{center}
\begin{asy}
pair T = dir(270);
pair S = dir(234);
pair N = dir(306);
pair A = dir(18);
pair C = dir(90);
pair K = dir(162);
pair U = incenter(S, N, C);

draw(CP(T, U), deepgreen);
draw(unitcircle, grey);
draw(S--N--A--C--K--cycle, grey);
draw(S--T--N, deepgreen);
draw(C--T, deepgreen);

draw(S--A, blue+1);
draw(N--K, red+1);

dot("$T$", T, dir(T));
dot("$S$", S, dir(190));
dot("$N$", N, dir(350));
dot("$A$", A, dir(A));
dot("$C$", C, dir(C));
dot("$K$", K, dir(K));
dot("$U$", U, dir(300));

/* TSQ Source:

T = dir 270
S = dir 234 R190
N = dir 306 R350
A = dir 18
C = dir 90
K = dir 162
U = incenter S N C R300

CP T U deepgreen
unitcircle grey
S--N--A--C--K--cycle grey
S--T--N deepgreen
C--T deepgreen

S--A blue+1
N--K red+1

*/
\end{asy}
\end{center}

(For a symmetric reason, the point $U$ also lies on line $SA$.)

\subsection*{Solution G2}
The hidden collinearity is that $G$, $E$, $R$ are collinear, which we prove now.

Evidently, the point $R$ is the one such that $ABCR$ is an isosceles trapezoid.
There is a negative homothety (with ratio $-2$) which sends
the nine-point circle to the $ABCR$, which maps $D$ to $A$ and $E$ to $R$.

\begin{center}
\begin{asy}
pair A = dir(110);
pair B = dir(210);
pair C = dir(330);
pair D = midpoint(B--C);
pair E = foot(A, B, C);
pair O = origin;
pair G = centroid(A, B, C);
pair R = B*C/A;
draw(E--R, deepgreen);
pair U = -R;
draw(A--D, grey);
draw(R--U, red+1);
draw(R--E, blue+1);
draw(unitcircle, grey);
draw(A--B--C--cycle, black);

dot("$A$", A, dir(A));
dot("$B$", B, dir(B));
dot("$C$", C, dir(C));
dot("$D$", D, dir(D));
dot("$E$", E, dir(E));
dot("$O$", O, dir(315));
dot("$G$", G, dir(G));
dot("$R$", R, dir(R));
dot("$U$", U, dir(U));

/* TSQ Source:

A = dir 110
B = dir 210
C = dir 330
D = midpoint B--C
E = foot A B C
O = origin R315
G = centroid A B C
R = B*C/A
E--R deepgreen
U = -R
A--D grey
R--U red+1
R--E blue+1
unitcircle grey
A--B--C--cycle black

*/
\end{asy}
\end{center}

(As for the originally posed problem,
the point $U$ is the reflection of the orthocenter
over line $BC$, which is known to be the antipode of $R$.)

\subsection*{Solution G3}
The hidden collinearity in the picture is that $H$, $E$, $L$ are collinear,
which we prove now.

\begin{center}
\begin{asy}
pair A = dir(110);
pair B = dir(210);
pair C = dir(330);
pair V = foot(C, A, B);
pair U = foot(B, C, A);
pair G = centroid(A, B, C);
pair H = orthocenter(A, B, C);
pair K = origin;
pair D = foot(A, B, C);
pair E = midpoint(B--C);
pair N = extension(U, V, B, C);
pair L = -A;
pair M = foot(A, L, N);
draw(A--B--C--cycle, black);
draw(H--K, blue+1);
draw(H--L, red+1);
draw(A--D, grey);
draw(A--E, grey);
draw(N--U--B, grey);
draw(N--L, grey);
draw(N--B, grey);
draw(unitcircle, grey);
draw(circumcircle(M, D, E), dashed+deepgreen);
draw(circumcircle(U, V, D), deepgreen);

dot("$A$", A, dir(A));
dot("$B$", B, dir(225));
dot("$C$", C, dir(315));
dot("$V$", V, dir(160));
dot("$U$", U, dir(45));
dot("$G$", G, dir(235));
dot("$H$", H, dir(H));
dot("$K$", K, dir(45));
dot("$D$", D, dir(D));
dot("$E$", E, dir(E));
dot("$N$", N, dir(N));
dot("$L$", L, dir(315));
dot("$M$", M, dir(205));

/* TSQ Source:

A = dir 110
B = dir 210 R225
C = dir 330 R315
V = foot C A B R160
U = foot B C A R45
G = centroid A B C R235
H = orthocenter A B C
K = origin R45
D = foot A B C
E = midpoint B--C
N = extension U V B C
L = -A R315
M = foot A L N R205
A--B--C--cycle black
H--K blue+1
H--L red+1
A--D grey
A--E grey
N--U--B grey
N--L grey
N--B grey
unitcircle grey
circumcircle M D E dashed deepgreen
circumcircle U V D deepgreen

*/
\end{asy}
\end{center}

Consider the circumcircles of $UVDE$ (nine-point circle),
$ABC$, and the triangle $MDE$.
The pairwise radical axii must be concurrent;
note that line $DE$ is one radical axis.
On the other hand, since $BVUC$ is a cyclic quadrilateral, we know
\[ ND \cdot NE = NU \cdot NV = NB \cdot NC  \]
so the point $N$ is the radical center, since it also lies on line $DE$.

In other words, the line $MN$ must pass through the second intersection
point of $(MDE)$ and $(ABC)$.
Hence $MDEL$ is cyclic.
But since $\angle AMN = 90\dg$, it follows $L$ must actually coincide
with the antipode of the point $A$.
It's well known the reflection of $H$ across $E$
is the desired antipode, so we're done.

(As for the originally proposed question,
$K$ is now the circumcenter of $ABC$
so the line $HKG$ is simply the Euler line.)

\subsection*{Solution G4}
The hidden collinearity in the picture is that $E$, $S$, $P$ are collinear.

\begin{center}
\begin{asy}
pair A = dir(110);
pair B = dir(210);
pair C = dir(330);
pair I = incenter(A, B, C);
pair D = foot(I, B, C);
pair E = foot(I, C, A);
pair F = foot(I, A, B);
pair L = midpoint(A--B);
pair N = midpoint(A--C);
pair M = midpoint(B--C);
pair R = extension(L, N, E, F);
pair S = extension(M, L, D, F);
pair P = extension(E, S, D, R);
pair U = extension(R, S, A, B);
draw(A--B--C--cycle, grey);
draw(circumcircle(D, E, F), grey);
draw(L--N, grey);
draw(D--S--M, deepgreen);
draw(E--S, red+1);
draw(R--S, blue+1);
draw(D--E--F--cycle, grey);
draw(E--F, grey);
draw(D--P, grey);
draw(U--P, grey);

dot("$A$", A, dir(A));
dot("$B$", B, dir(B));
dot("$C$", C, dir(C));
dot("$D$", D, dir(D));
dot("$E$", E, dir(E));
dot("$F$", F, dir(200));
dot("$L$", L, dir(100));
dot("$N$", N, dir(N));
dot("$M$", M, dir(M));
dot("$R$", R, dir(315));
dot("$S$", S, dir(S));
dot("$P$", P, dir(60));
dot("$U$", U, dir(205));

/* TSQ Source:

A = dir 110
B = dir 210
C = dir 330
I := incenter A B C
D = foot I B C
E = foot I C A
F = foot I A B R200
L = midpoint A--B R100
N = midpoint A--C
M = midpoint B--C
R = extension L N E F R315
S = extension M L D F
P = extension E S D R R60
U = extension R S A B R205
A--B--C--cycle grey
circumcircle D E F grey
L--N grey
D--S--M deepgreen
E--S red+1
R--S blue+1
D--E--F--cycle grey
E--F grey
D--P grey
U--P grey

*/
\end{asy}
\end{center}

We will need that the Feurerbach hyperbola is the rectangular circumhyperbola
of triangle $ABC$ which passes through the point $I$,
and that the center of this hyperbola is the Feurerbach point $P$.
Then the conclusion follows by the
\href{https://mathworld.wolfram.com/FonteneTheorems.html}{First Fonten\'{e} Theorem}:
the points $R = EF \cap NL$, $S = FD \cap LM$, $T = DE \cap MN$ (not shown)
should have lines $DR$, $ES$, $FT$ meeting at the point $P$.

(For the actually posed question,
note first by Brokard's theorem on $PFDE$ that the points
$U = PP \cap FF$, $R = PD \cap EF$, $S = PE \cap DF$ are collinear.
Incidentally, the point $C = DD \cap EE$ lies on this line too.)


\subsection*{Solution G5}
The hidden collinearity in the picture is that $V$, $M$, $N$ are collinear,
which we prove now.
(Technically speaking, $OGH$ and $COX$ are also collinear,
but there is no country with code using those triples of letters.)

\begin{center}
\begin{asy}
pair A = dir(110);
pair B = dir(210);
pair C = dir(330);
pair M = midpoint(A--B);
pair N = midpoint(A--C);
pair P = midpoint(B--C);
pair H = orthocenter(A, B, C);
pair O = circumcenter(A, B, C);
pair G = centroid(A, B, C);
pair E = C+2*foot(O,M,H);
pair D = B+2*foot(O,N,H);
pair V = extension(M, N, D, E);
draw(V--N, red+1);
draw(V--A, grey);
draw(unitcircle, grey);
draw(A--B--C--cycle, black);
draw(circumcircle(M, N, D), deepgreen);
draw(M--E, grey);
draw(N--D, grey);
draw(V--E, deepgrey);
pair X = -C;
pair W = -A+2*foot(A, O, V);
pair L = B*C/A;
draw(W--L, lightgrey);
draw(X--E, blue+1);

dot("$A$", A, dir(A));
dot("$B$", B, dir(270));
dot("$C$", C, dir(C));
dot("$M$", M, dir(135));
dot("$N$", N, dir(N));
dot("$P$", P, dir(P));
dot("$H$", H, dir(270));
dot("$O$", O, dir(315));
dot("$G$", G, dir(270));
dot("$E$", E, dir(E));
dot("$D$", D, dir(D));
dot("$V$", V, dir(V));
dot("$X$", X, dir(X));
dot("$W$", W, dir(W));
dot("$L$", L, dir(L));

/* TSQ Source:

A = dir 110
B = dir 210 R270
C = dir 330
M = midpoint A--B R135
N = midpoint A--C
P = midpoint B--C
H = orthocenter A B C R270
O = circumcenter A B C R315
G = centroid A B C R270
E = C+2*foot(O,M,H)
D = B+2*foot(O,N,H)
V = extension M N D E
V--N red+1
V--A grey
unitcircle grey
A--B--C--cycle black
circumcircle M N D deepgreen
M--E grey
N--D grey
V--E deepgrey
X = -C
W = -A+2*foot A O V
L = B*C/A
W--L lightgrey
X--E blue+1

*/
\end{asy}
\end{center}

To prove this, note that a negative inversion at $H$
carrying the nine-point circle to the circumcircle
will map $M$ to $E$ and $N$ to $D$,
so that the points $M$, $N$, $E$, $D$ are cyclic.
In addition, the circumcircle of $AMN$ is tangent to
the circumcircle of $ABC$ at the point $A$.
So considering the radical axis of the circles $AMN$, $ABC$, $MNDE$
shows that the tangent at $A$, the line $DE$, and the line $MN$ concur --- at $V$.
Hence $V$, $M$, $N$ are collinear.

(For the originally suggested problem,
one can note that $VW$ is tangent to the circumcircle of $ABC$
and then, following the approach of IMO Shortlist 2011 G4,
show that $L$ is the point for which $ABCL$ is an isosceles trapezoid.
The contrived angle condition that $X$ is the $C$-antipode, as needed.)


\section{Number Theory}
\subsection*{Solution NT1}
This is a classical variant of the Frobenius coin problem.
We need $\gcd(m,n) = 1$ or else any integer
relatively prime to $\gcd(m,n)$ is not expressible.

It is known that in when $\gcd(m,n) = 1$
then the number of integers which can't be formed is
\[ \frac{mn-m-n}{2}. \]
So we wish to solve
\[ \frac{mn-m-n}{2} = 4161
	\iff (m-1)(n-1) = mn - m - n + 1 = 8322. \]
Since $8322 = 2 \cdot 3 \cdot 19 \cdot 73$,
the requested pair is $(m-1, n-1) = (73, 114)$
or \[ (m,n) = (74,115). \]

\subsection*{Solution NT2}
The condition rewrites as
\[ (7m+2n)^2 + (3m+n)^2 = 193. \]
The number $193$ is prime and can be expressed as the sum of two squares
in exactly one way, $193 = 12^2+7^2$.
So we want either $7m+2n = \pm 12$ and $3m+n = \pm 7$,
or $7m+2n = \pm 7$ and $3m+n = \pm 12$.

We exhaust all cases below:
\begin{center}
\begin{tabular}[h]{ccc}
	\multicolumn{2}{c}{System of equations} & Solution \\ \hline
	$7m+2n = \pm12$ & $3m+n = \pm7$ & $(m,n) = \left( \mp 2, \pm 13 \right)$ \\
	$7m+2n = \pm12$ & $3m+n = \mp7$ & $(m,n) = \left( \pm 26, \mp 85 \right)$ \\
	$7m+2n = \pm7$ & $3m+n = \pm12$ & $(m,n) = \left( \mp17, \pm63 \right)$ \\
	$7m+2n = \pm7$ & $3m+n = \mp12$ & $(m,n) = \left( \pm31, \mp105 \right)$.
\end{tabular}
\end{center}
For $n$ to be as large as possible, we want the pair $(-31, 105)$.

\subsection*{Solution NT3}
Obviously we need $m \ge 0$ or the LHS is not an integer.
When $m = 0$ or $m = 1$ there is no solution so we assume $m \ge 2$.

First, taking modulo $8$, we have $3^m \equiv (2n-3)^2 \equiv 1 \pmod 8$,
integer $m$ must be even.
When considering $8(m-1)^2 = 3^m - (2n-3)^2$,
as both terms are positive
we conclude that both factors are positive,
i.e.\ that $3^{m/2} > |2n-3|$.
Therefore, we obtain the crude estimate
\[ 8(m-1)^2 \ge 3^m - (3^{m/2}-1)^2 = 2 \cdot 3^{m/2} - 1. \]
which in fact can only hold for $m \le 10$.

We now check all the cases:
\begin{center}
\begin{tabular}[h]{cccc}
	$m$ & $3^m - 8(m-1)^2$ & $2n-3$ & $n$ \\ \hline
	$m=2$ & $1$ & $1$ & $n=1$ or $n=2$ \\
	$m=4$ & $9$ & $3$ & $n=0$ or $n=3$ \\
	$m=6$ & $529$ & $23$ & $n=-10$ or $n=13$ \\
	$m=8$ & $6169$ & n/a & n/a \\
	$m=10$ & $58401$ & n/a & n/a. \\
\end{tabular}
\end{center}
This gives $(6,-10)$ as the pair with $m-n$ as large as possible.

\subsection*{Solution NT4}
By AM-GM, we have
\[ f(\pi) \ge \sqrt[3]{9!} \approx 213.98. \]
So we have $n \ge 214$ and $9m \ge 216$.
Both can be achieved:
\begin{align*}
	1 \cdot 8 \cdot 9 + 2 \cdot 5 \cdot 7 + 3 \cdot 4 \cdot 6 = 214 \\
	1 \cdot 8 \cdot 9 + 2 \cdot 6 \cdot 7 + 3 \cdot 4 \cdot 5 = 216.
\end{align*}
Hence the answer $(24, 214)$.

\end{document}
